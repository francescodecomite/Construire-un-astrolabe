
\documentclass{beamer}
%\documentclass[handout]{beamer}
\usepackage{pgfpages}
%\setbeameroption{show notes on second screen=left}
%\setbeameroption{hide notes}
%\setbeameroption{show only  notes}

\usepackage{xcolor}
\usepackage[utf8]{inputenc}
\usepackage[francais]{babel}
\usepackage{multimedia}
%comment
\setbeamertemplate{navigation symbols}{} 
 \usetheme{Warsaw}
\setbeamercovered{highly dynamic}

\usepackage{multimedia}

%\usepackage{pgfpages}
%\mode<handout>{\pgfpagesuselayout{4 on 1}[a4paper,landscape,border shrink=5mm]}
\mode<handout>{\pgfpagesuselayout{8 on 1}[a4paper,border shrink=5mm]}
\mode<handout>{\setbeamercolor{background canvas}{bg=black!5}}


\pgfdeclareimage[height=0.9\textheight]{lautrec}{images/lautrec}
\pgfdeclareimage[height=0.9\textheight]{rouage}{images/rouage}
\pgfdeclareimage[height=0.9\textheight]{lait}{images/lait}
\pgfdeclareimage[height=0.9\textheight]{gimp}{images/gimp}
\pgfdeclareimage[height=0.9\textheight]{vachequirit}{images/vachequirit}
\pgfdeclareimage[height=0.9\textheight]{rubik}{images/rubik}
\pgfdeclareimage[height=0.9\textheight]{ring}{images/ring}
\pgfdeclareimage[height=0.9\textheight]{menger6}{images/menger6}
\pgfdeclareimage[height=0.9\textheight]{blender}{images/blender}
\pgfdeclareimage[height=0.9\textheight]{baseball}{images/baseball}
\pgfdeclareimage[height=0.9\textheight]{mandel1}{images/mandel1}
\pgfdeclareimage[height=0.9\textheight]{mandel2}{images/mandel2}
\pgfdeclareimage[height=0.9\textheight]{mandel3}{images/mandel3}
\pgfdeclareimage[height=0.9\textheight]{tetra}{images/tetra}
\pgfdeclareimage[height=0.9\textheight]{sierpinski}{images/sierpinski}
\pgfdeclareimage[height=0.9\textheight]{octa}{images/octa}
\pgfdeclareimage[height=0.9\textheight]{icosa}{images/icosi}
\pgfdeclareimage[height=0.9\textheight]{dodeca}{images/dodeca}
\pgfdeclareimage[height=0.9\textheight]{archi}{images/archi}
\pgfdeclareimage[height=0.9\textheight]{archi2}{images/archi2}
\pgfdeclareimage[height=0.9\textheight]{platon}{images/platon}
\pgfdeclareimage[height=0.9\textheight]{menger}{images/menger}
\pgfdeclareimage[height=0.8\textheight]{cutmenger}{images/cutmenger}
\pgfdeclareimage[height=0.9\textheight]{cutdodeca}{images/cutdodeca}
\pgfdeclareimage[height=0.9\textheight]{apollon}{images/Apollon}
\pgfdeclareimage[height=1.3\textheight]{calcul}{images/calcul}
\pgfdeclareimage[height=0.9\textheight]{chou}{images/chou}
\pgfdeclareimage[height=0.9\textheight]{fougere}{images/fougere}
\pgfdeclareimage[height=0.9\textheight]{fern}{images/fern}
\pgfdeclareimage[height=0.9\textheight]{foug}{images/foug}
\pgfdeclareimage[width=\textwidth]{cantor}{images/cantor}
\pgfdeclareimage[height=0.9\textheight]{koch}{images/koch}
\pgfdeclareimage[height=0.9\textheight]{hilbert}{images/hilbert}
\pgfdeclareimage[height=0.8\textheight]{mandelbrot}{images/mandelbrot}
\pgfdeclareimage[height=0.8\textheight]{brooksmatelski1978}{images/Mandel}
\pgfdeclareimage[height=0.9\textheight]{m1}{images/m1}
\pgfdeclareimage[height=0.9\textheight]{m2}{images/m2}
\pgfdeclareimage[height=0.9\textheight]{m3}{images/m3}
\pgfdeclareimage[height=0.9\textheight]{m4}{images/m4}
\pgfdeclareimage[height=0.9\textheight]{bulb1}{images/bulb1}
\pgfdeclareimage[height=0.9\textheight]{bulb2}{images/bulb2}
\pgfdeclareimage[height=0.9\textheight]{bulb3}{images/bulb3}
\pgfdeclareimage[height=0.8\textheight]{denevan}{images/denevan}



\newtheorem{fdefinition}[theorem]{Définition}
\newtheorem{Remarque}[theorem]{Remarque}
\newtheorem{Remarques}[theorem]{Remarques}
%\newtheorem{Theoreme}[theorem]{Théorème}
\newtheorem{notation}[theorem]{Notation}
  \title[Construire un astrolabe]{Construire un astrolabe}
  
  \author{Francesco De Comité}\institute{Ex-Université de Lille \\ Faculté des Sciences et Technologies}
%\titlegraphic{\includegraphics[width=3.5cm]{images/logo.jpg}\hfill}

%\titlegraphic{\ \vfill\vfill\hfill\includegraphics[width=3.5cm]{images/logo.jpg}\hfill\hfill\hfill\hfill \raisebox{0.25cm}{\includegraphics[width=1.1cm]{images/logoCNRS.jpg}}\hfill\hfill}


%\logo{\includegraphics[height=0.5cm]{lille1.png}}

\setbeamertemplate{part page}{
        \begin{beamercolorbox}[sep=8pt,center,wd=\textwidth]{part title}
            \usebeamerfont{part title}\insertpart\par
        \end{beamercolorbox}
}


%%%%%%%%%%%%%%%

\setbeamertemplate{part page}{
        \begin{beamercolorbox}[sep=8pt,center,wd=\textwidth, rounded=true,shadow=true]{part title}
            \usebeamerfont{part title}\insertpart\par
        \end{beamercolorbox}
}

%\definecolor{Universite}{RGB}{181,0,103}
\definecolor{Espace}{RGB}{0,37,115}

   \setbeamercolor{item projected}{bg=Espace}

\setbeamercolor{title}{bg=Espace}
\setbeamercolor{part title}{bg=Espace}
\setbeamercolor{frametitle}{bg=Espace}
%\setbeamercolor{structure}{bg=Universite}
\setbeamercolor{block title}{bg=Espace}
%\setbeamercolor{title}{fg=white,bg=!181!0!103}

\setbeamercolor{section in head/foot}{bg=Espace}

\makeatletter
\setbeamertemplate{footline}
{
  \leavevmode%
  \hbox{%
  \begin{beamercolorbox}[wd=.25\paperwidth,ht=2.25ex,dp=1ex,center]{section in head/foot}%
    \usebeamerfont{author in head/foot}\insertshortauthor
  \end{beamercolorbox}%
  \begin{beamercolorbox}[wd=.60\paperwidth,ht=2.25ex,dp=1ex,center]{section in head/foot}%
    \usebeamerfont{title in head/foot}\insertshorttitle
  \end{beamercolorbox}%
  \begin{beamercolorbox}[wd=.15\paperwidth,ht=2.25ex,dp=1ex,right]{section in head/foot}%
   % \usebeamerfont{date in head/foot}\insertshortdate{}\hspace*{2em}
    %\insertframenumber{} / \inserttotalframenumber\hspace*{2ex} 
	%\hfill\raisebox{-0.11cm}{\includegraphics[width=0.7cm]{images/logo.jpg}}
  \end{beamercolorbox}}%
  \vskip0pt%
}
\makeatother


%%%%%%%%%%%%%%

\begin{document}

%%%%%%%%%%%%%%%%%%%%%%%%%%%%
 \begin{frame}
\titlepage
\note[item]{bla bla}
   \end{frame}
%%%%%%%%%%%%%%%%%%%%%%%%%%%%%%%

%%%%%%%%%%%%%%%%%%%%%%%%%%%%%%%%%%%
  
 \begin{frame}\frametitle{Généralités}

\begin{block}{Qui suis-je ? }
\begin{itemize}
\item Maître de conférences en informatique (retraité) \\ à l'Université de Lille
\item Enseignements en programmation, maths pour l'informatique, logique, intelligence artificielle. 
\item Recherche en IA, puis en Art, Mathématiques, Informatique. 
\item \begin{block}{Donner une réalité à des objets mathématiques. }
	\begin{itemize}
	\item Logiciels adéquats (Blender, Rhino, Inkscape)
	\item Langages de description (SVF, STL)
	\item Outils de conception : imprimantes 3D, découpe laser, atelier {\it ``standard''}.
	\end{itemize}
	\end{block}
\end{itemize}
\end{block}

   \end{frame}


%%%%%%%%%%%%%%%%%%%%%%%%%%%%%%%%%%%
  
 \begin{frame}\frametitle{Pourquoi construire un astrolabe ?}
 
 \begin{block}{Parce qu'on peut !}
 \begin{itemize}
 \item Des documents très bien écrits (Michel Dumas).
 \item Des logiciels.
 \item Des machines.
 \item Des gens. 
 \end{itemize}
 
 
 
 \end{block}
 
\end{frame}


\begin{frame}\frametitle{Le plan}
\begin{block}

\end{block}

   \end{frame}
   
\begin{frame}\frametitle{}

   \end{frame}
   
\begin{frame}\frametitle{}

   \end{frame}
   
\begin{frame}\frametitle{}

   \end{frame}
 \end{document}

 \begin{frame}\frametitle{Carrière}
 \begin{itemize}
 \item Nommé maître de conférences en 1984
 \end{itemize}
\begin{block}{Domaines de recherche  }
\begin{itemize}
\item Informatique théorique ({\em systèmes de réécriture})
\item Intelligence artificielle (apprentissage automatique : réseaux de neurones \dots)
\item Art et ordinateur.
\end{itemize}
\end{block}


   \end{frame}

 \begin{frame}\frametitle{Art et ordinateur}
 \begin{block}{La démarche}
 \begin{itemize}
 \item Représenter des concepts mathématiques sous formes de dessins, de scènes, d'objets. 
 \item Utiliser des outils informatiques si possibles libres, open-source, gratuits. 
 \item Expliquer la méthode utilisée, diffuser ce que j'ai appris. 
 \end{itemize}
\end{block}
\end{frame}

\begin{frame}\frametitle{Outils}

   
\begin{block}{Les logiciels}
\begin{itemize}
\item Gimp : Manipulation d'images ($\approx$ Photoshop)
\item Povray : Création de scènes. 
\item Blender : Modélisation 3D. Produit dessins et objets 3D
\item Rhino + Grasshopper: Comme Blender, mais payant (versions d'évaluations limitées)
\item Langages de programmation :Java, Python
\item Inkscape : découpe laser. 
\end{itemize}
\end{block}
\end{frame}



\begin{frame}\frametitle{Gimp : Manipulation d'image}
\begin{center}
\pgfuseimage{rouage}
\end{center}

 \end{frame}
   
\begin{frame}\frametitle{Gimp : Manipulation d'images}
\begin{center}
\pgfuseimage{lautrec}
\end{center}
   \end{frame}
   
\begin{frame}\frametitle{Gimp : Manipulation d'images}
\begin{center}
\pgfuseimage{vachequirit}
\end{center}
   \end{frame}
\begin{frame}\frametitle{Gimp : Programmation , pixels}
\begin{center}
\pgfuseimage{gimp}
\end{center}
   \end{frame}
   

\begin{frame}\frametitle{Gimp : Programmation , pixels}
\begin{center}
\pgfuseimage{lait}
\end{center}
   \end{frame}
   

   

\begin{frame}\frametitle{Povray}
\begin{block}{Présentation}
\begin{itemize}
\item Povray permet de décrire des {\em scènes} : 
\begin{itemize}
\item Placer des objets (sphères, cylindres, surfaces \dots)
\item Définir les matériaux : couleur, reflets, transparence. 
\item Placer des éclairages
\item Placer une caméra.
\end{itemize}
\item Et puis on prend une photo. 
\end{itemize}
\end{block}
\begin{itemize}
\item Un plus : Povray contient un langage de programmation {\em standard}
\item On peut utiliser des boucles, et créér un grand nombre d'objets. 
\end{itemize}

   \end{frame}
   

\begin{frame}\frametitle{Povray : création de scène}
\begin{center}
\pgfuseimage{rubik}
\end{center}
   \end{frame}
   

\begin{frame}\frametitle{Povray : Programmation}
\begin{center}
\pgfuseimage{ring}
\end{center}
   \end{frame}
   

\begin{frame}\frametitle{Povray : Programmation}
\begin{center}
\pgfuseimage{menger6}
\end{center}
   \end{frame}
   

\begin{frame}\frametitle{Blender}
\begin{block}{Caractéristiques}
\begin{itemize}
\item Blender permet de décrire des scènes, comme Povray. 
\item Mais les objets créés peuvent être exportés sous formes de fichiers 3D
\item La création d'images 2D est plus laborieuse qu'avec Povray
\end{itemize}
\end{block}

   \end{frame}
   

\begin{frame}\frametitle{Blender}
\begin{center}
\pgfuseimage{blender}
\end{center}
   \end{frame}
   

\begin{frame}\frametitle{Blender}
\begin{center}
\pgfuseimage{baseball}
\end{center}
   \end{frame}
   
\part{Les fractales dans mon travail}

\part{Les polièdres fractals}

\begin{frame}\frametitle{Mandelbrot années 80}
\begin{center}
\pgfuseimage{mandel1}
\end{center}
   \end{frame}
   
\begin{frame}\frametitle{Mandelbrot années 80}
\begin{center}
\pgfuseimage{mandel2}
\end{center}
   \end{frame}   
   
   
\begin{frame}\frametitle{Povray}
\begin{center}
\pgfuseimage{mandel3}
\end{center}
   \end{frame}
   
   \begin{frame}\frametitle{Povray}
\begin{center}
\pgfuseimage{ring}
\end{center}
   \end{frame}
   

\begin{frame}\frametitle{Povray}
\begin{center}
\pgfuseimage{menger6}
\end{center}
   \end{frame}
   
   \begin{frame}\frametitle{Blender}
\begin{center}
\pgfuseimage{menger}
\end{center}
   \end{frame}
   
\begin{frame}\frametitle{Povray}
\begin{center}
\pgfuseimage{sierpinski}
\end{center}
   \end{frame}
   
\begin{frame}\frametitle{Povray}
\begin{center}
\pgfuseimage{tetra}
\end{center}
   \end{frame}
   
  \begin{frame}\frametitle{Solides platoniciens}
\begin{center}
\pgfuseimage{platon}
\end{center}
   \end{frame}
    
   
\begin{frame}\frametitle{Icosaèdre (20 faces)}
\begin{center}
\pgfuseimage{icosa}
\end{center}
   \end{frame}
   
\begin{frame}\frametitle{Octaèdre (8 faces)}
\begin{center}
\pgfuseimage{octa}
\end{center}
   \end{frame}
   
\begin{frame}\frametitle{Dodecaèdre (12 faces)}
\begin{center}
\pgfuseimage{dodeca}
\end{center}
   \end{frame}
   
\begin{frame}\frametitle{Solides archimédiens}
\begin{center}
\pgfuseimage{archi2}
\end{center}
   \end{frame}
   
\begin{frame}\frametitle{Solides archimédiens}
\begin{center}
\pgfuseimage{archi}
\end{center}
   \end{frame}
   
\begin{frame}\frametitle{Un peu de reflexion}
\begin{itemize}
\item Pourquoi le tétraèdre est plus {\em fin} que les autres solides platoniciens ? 
\item Pourquoi le dodécaèdre est-il moint précis ? 
\item Quand on passe d'un niveau au suivant dans un cube de Menger, par combien multplie-t-on le nombre de cubes élémentaires ? 
\item Que dire de la masse d'une éponge de Menger ? 
\item Que dire de sa surface ? $\longleftarrow$ ça, c'est dur \dots
\end{itemize}


   \end{frame}
   
\begin{frame}\frametitle{Quand l'informatique amène des découvertes}
\begin{center}
\pgfuseimage{cutmenger}
\end{center}
\begin{itemize}
\item A ma connaissance, ce résultat était inconnu (et c'est quelque chose de surprenant) avant que quelqu'un ne le programme. 
\end{itemize}
   \end{frame}
   
   \begin{frame}\frametitle{Couper un dodécahèdre}
\begin{center}
\pgfuseimage{cutdodeca}
\end{center}
   \end{frame}
   
 \part{Les empilements de cercles}
   
\begin{frame}\frametitle{La baderne d'Apollonius}
\begin{center}
\pgfuseimage{apollon}
\end{center}
   \end{frame}
   
\begin{frame}\frametitle{La baderne d'Apollonius}
\begin{block}{Petites questions}
\begin{enumerate}
\item Comment est construite cette image ? 
\item Si le grand cercle extérieur a un rayon de $R$, quel est le rayon des trois cercles moyens ? 
\item Le rayon du petit cercle interne ? 
\item Le rayon des trois cercles qui ont le troisième rayon en ordre de taille décroissante ? 
\end{enumerate}



\end{block}

   \end{frame}
   
\begin{frame}\frametitle{La baderne d'Apollonius}

\begin{block}{Petites réponses}
\begin{enumerate}
\item \dots
\item Voir le transparent suivant. 
\item Pareil
\item ça, c'est très dur \dots

\end{enumerate}


\end{block}

   \end{frame}
   
\begin{frame}\frametitle{Réponses}
\begin{center}
\pgfuseimage{calcul}
\end{center}
   \end{frame}
   
\begin{frame}\frametitle{Réponses}
\begin{itemize}
\item $R=R_1+2\times R_2$
\item Un triangle rectangle avec des angles de 30,60 et 90 degrés
\item $ \cos(30)= \frac{R_2} {(R_1+R_2)}=\frac{\sqrt(3)}{2}$
\item $R1=R-2\times R2$
\item $R_1+R_2=R-R_2$
\item Moralité : on connaît $R_2$ en fonction de $R$. 
\item On peut en déduire $R_1$.
\item En gros, de la trigonométrie de base. On doit pouvoir s''en tirer avec juste le théorème de Pythagore. 
\end{itemize}
\begin{block}{Spoiler}
On reviendra sur ce point un peu plus loin, quand on prlera des fractales réalisables. 
\end{block}



   \end{frame}
   
   \part{Un catalogue }
   
\begin{frame}\frametitle{Catalogue}
\begin{center}
\pgfuseimage{chou}
\end{center}
   \end{frame}
   
\begin{frame}\frametitle{Fougère}
\begin{center}
\pgfuseimage{fougere}
\end{center}
   \end{frame}
   
\begin{frame}\frametitle{Fougère}
\begin{center}
\pgfuseimage{fern}
\end{center}
   \end{frame}
   
\begin{frame}\frametitle{Fougère}
\begin{center}
\pgfuseimage{foug}
\end{center}
   \end{frame}
   
\begin{frame}\frametitle{La poussière de Cantor}
\begin{center}
\pgfuseimage{cantor}
\end{center}
   \end{frame}
   
\begin{frame}\frametitle{La courbe de Von Koch}
\begin{center}
\pgfuseimage{koch}
\end{center}
   \end{frame}
   
\begin{frame}\frametitle{La courbe de Hilbert}
\begin{center}
\pgfuseimage{hilbert}
\end{center}
   \end{frame}
   
   
   
    \begin{frame}\frametitle{L'ensemble de Mandelbrot : définition}
\begin{block}{Une formulation simple}
Soit $c$ un point du plan complexe. 
\begin{equation*}
\begin{split}
z_0&=0\\
z_{n+1}&=z_n^2+c
\end{split}
\end{equation*}
\end{block}
\begin{block}{Représentation graphique}
\begin{itemize}
\item Si la suite $(z_n)$ diverge après $n_0$ itérations, dessiner le point $c$ avec une couleur correspondant à $n_0$. 
\item Sinon, afficher un point noir en $c$.
\end{itemize}
\end{block}
  \note[item]{La fractale la plus célèbre est sans doute l'ensemble de Mandelbrot. Elle est liée à la suite définie de manière récursive pour chaque point $c$ du plan complexe par la formule ci-contre. Pour certains points, cette suite reste `sage'. Pour d'autres, les $z_n$ s'échappent vers l'infini.}

   \end{frame}
 

\begin{frame}\frametitle{L'ensemble de Mandelbrot}
\begin{center}
{\pgfuseimage{mandelbrot}}
\end{center}
\vfill
\note[item]{Pour construire ce dessin, on opère de la façon suivante: 
\begin{itemize}
\item on prend un point du plan (le point c), et on calcule la suite des $z_n$. 
\item Les images successives du point vont se promener dans le plan. 
\item Quelquefois, ces orbites restent dans le voisinage de l'origine (en gros, sur l'image). 
\item Pour d'autres points, ils s'éloignent à l'infini après un certains nombre de sauts. 
\end{itemize}
On colorie en noir les points qui restent dans le voisinage, tandis que ceux qui s'éloignent ont une couleur qui dépend de l'étape à laquelle ils se sont éloignés.
}

\hfill{\tiny  Wolfgang Beyer, Wikipedia}

   \end{frame}
   
     
 \begin{frame}\frametitle{L'ensemble de Mandelbrot : première image}
 \begin{center}
{\pgfuseimage{brooksmatelski1978}}
\end{center}
\vfill
\hfill{\tiny Brooks \& Matelski 1978}
\note[item]{La règle est simple, le résultat est un dessin très compliqué, où l'on retrouve plein de copies du dessin initial à tous les grossissements}
\note[item]{On peut zoomer, voler au-dessus de l'ensemble pour explorer des zones, comme si on survolait une planète ou un continent inconnu. C'est l'exploration visuelle, plutôt que la reflection et le calcul, qui permet d'en découvrir les plus beaux `paysages'}
\note[item]{ Voici le premier `dessin' publié de l'ensemble de Mandelbrot, je n'ai pas cherché combien il avait coûté en temps de calcul\dots}
   \end{frame}
      

   
 
   
\begin{frame}\frametitle{Zoom }
 \begin{center}
\pgfuseimage{m1}
\end{center}
   \end{frame}
   \begin{frame}\frametitle{Zoom}
\begin{center}
\pgfuseimage{m2}
\end{center}
   \end{frame}
   
\begin{frame}\frametitle{Zoom}
\begin{center}
\pgfuseimage{m3}
\end{center}
   \end{frame}
   \begin{frame}\frametitle{Zoom}
\begin{center}
\pgfuseimage{m4}
\end{center}
   \end{frame}
  
     \begin{frame}\frametitle{L'ensemble de Mandelbrot}
   \begin{block}{Moralité}
\begin{itemize}
\item Sans ordinateur, on ne connaîtrait pas la forme de l'ensemble de Mandelbrot. 
\item Sans ordinateur, on ne pourrait pas zoomer et explorer cet ensemble. 
\item Pour obtenir de belles images, il faut un ordinateur rapide et un programme efficace. 
\end{itemize}
\end{block}
   \end{frame}
   
\begin{frame}\frametitle{Mandelbulb : une généralisation en 3D}
\begin{center}
\pgfuseimage{bulb1}
\end{center}
   \end{frame}
   
   \begin{frame}\frametitle{Mandelbulb : une généralisation en 3D}
\begin{center}
\pgfuseimage{bulb2}
\end{center}
   \end{frame}
   
   \begin{frame}\frametitle{Mandelbulb : une généralisation en 3D}
\begin{center}
\pgfuseimage{bulb3}
\end{center}
   \end{frame}
   
   
   \begin{frame}\frametitle{Suite du programme et pistes}
   \begin{block}{Cercles tangents}
   \begin{itemize}
   \item En reprenant le théorème de Descartes, on peut penser à un procédé de création d'un ensemble de cercles tangents. 
   \item Si on utilise la version complexe, on peut penser à une installation géante. 
   \end{itemize}
   
  
   \end{block}


   \end{frame}


    \begin{frame}\frametitle{Perspectives et applications possibles}
\begin{center}
\pgfuseimage{denevan}

\end{center}
\hfill{ Jim Denevan}
\note[item]{Jim Denevan : j'ai déjà essayé de faire ça à la plage, mais il faut quand même avoir pas mal de metier\dots}
   \end{frame}
   








































\end{document}

\begin{frame}\frametitle{}

   \end{frame}
   
\begin{frame}\frametitle{}

   \end{frame}
   
\begin{frame}\frametitle{}

   \end{frame}
   
\begin{frame}\frametitle{}

   \end{frame}
   

