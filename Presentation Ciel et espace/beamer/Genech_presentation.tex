
\documentclass{beamer}
%\documentclass[handout]{beamer}
\usepackage{pgfpages}
%\setbeameroption{show notes on second screen=left}
%\setbeameroption{hide notes}
%\setbeameroption{show only  notes}

\usepackage{xcolor}
\usepackage[utf8]{inputenc}
\usepackage[francais]{babel}
\usepackage{multimedia}
%comment
\setbeamertemplate{navigation symbols}{} 
 \usetheme{Warsaw}
\setbeamercovered{highly dynamic}

\usepackage{multimedia}

%\usepackage{pgfpages}
%\mode<handout>{\pgfpagesuselayout{4 on 1}[a4paper,landscape,border shrink=5mm]}
\mode<handout>{\pgfpagesuselayout{8 on 1}[a4paper,border shrink=5mm]}
\mode<handout>{\setbeamercolor{background canvas}{bg=black!5}}


\pgfdeclareimage[height=0.9\textheight]{ask}{images/ask}
\pgfdeclareimage[height=0.9\textheight]{github}{images/github}
\pgfdeclareimage[height=0.9\textheight]{qrcode}{images/qrcode}
\pgfdeclareimage[height=0.9\textheight]{githubconso1}{images/githubconso1}
\pgfdeclareimage[height=0.9\textheight]{fichierssvg}{images/fichierssvg}
\pgfdeclareimage[height=0.9\textheight]{githubpro1}{images/githubpro1}
\pgfdeclareimage[height=0.85\textheight]{astrolabe1}{images/astrolabe1}
\pgfdeclareimage[height=0.85\textheight]{astrolabe2}{images/astrolabe2}
\pgfdeclareimage[height=0.9\textheight]{menger6}{images/menger6}
\pgfdeclareimage[height=0.9\textheight]{blender}{images/blender}
\pgfdeclareimage[height=0.9\textheight]{baseball}{images/baseball}
\pgfdeclareimage[height=0.9\textheight]{mandel1}{images/mandel1}
\pgfdeclareimage[height=0.9\textheight]{mandel2}{images/mandel2}
\pgfdeclareimage[height=0.9\textheight]{mandel3}{images/mandel3}
\pgfdeclareimage[height=0.9\textheight]{tetra}{images/tetra}
\pgfdeclareimage[height=0.9\textheight]{sierpinski}{images/sierpinski}
\pgfdeclareimage[height=0.9\textheight]{octa}{images/octa}
\pgfdeclareimage[height=0.9\textheight]{icosa}{images/icosi}
\pgfdeclareimage[height=0.9\textheight]{dodeca}{images/dodeca}
\pgfdeclareimage[height=0.9\textheight]{archi}{images/archi}
\pgfdeclareimage[height=0.9\textheight]{archi2}{images/archi2}
\pgfdeclareimage[height=0.9\textheight]{platon}{images/platon}
\pgfdeclareimage[height=0.9\textheight]{menger}{images/menger}
\pgfdeclareimage[height=0.8\textheight]{cutmenger}{images/cutmenger}
\pgfdeclareimage[height=0.9\textheight]{cutdodeca}{images/cutdodeca}
\pgfdeclareimage[height=0.9\textheight]{apollon}{images/Apollon}
\pgfdeclareimage[height=1.3\textheight]{calcul}{images/calcul}
\pgfdeclareimage[height=0.9\textheight]{chou}{images/chou}
\pgfdeclareimage[height=0.9\textheight]{fougere}{images/fougere}
\pgfdeclareimage[height=0.9\textheight]{fern}{images/fern}
\pgfdeclareimage[height=0.9\textheight]{foug}{images/foug}
\pgfdeclareimage[width=\textwidth]{cantor}{images/cantor}
\pgfdeclareimage[height=0.9\textheight]{koch}{images/koch}
\pgfdeclareimage[height=0.9\textheight]{hilbert}{images/hilbert}
\pgfdeclareimage[height=0.8\textheight]{mandelbrot}{images/mandelbrot}
\pgfdeclareimage[height=0.8\textheight]{brooksmatelski1978}{images/Mandel}
\pgfdeclareimage[height=0.9\textheight]{m1}{images/m1}
\pgfdeclareimage[height=0.9\textheight]{m2}{images/m2}
\pgfdeclareimage[height=0.9\textheight]{m3}{images/m3}
\pgfdeclareimage[height=0.9\textheight]{m4}{images/m4}
\pgfdeclareimage[height=0.9\textheight]{bulb1}{images/bulb1}
\pgfdeclareimage[height=0.9\textheight]{bulb2}{images/bulb2}
\pgfdeclareimage[height=0.9\textheight]{bulb3}{images/bulb3}
\pgfdeclareimage[height=0.8\textheight]{denevan}{images/denevan}



\newtheorem{fdefinition}[theorem]{Définition}
\newtheorem{Remarque}[theorem]{Remarque}
\newtheorem{Remarques}[theorem]{Remarques}
%\newtheorem{Theoreme}[theorem]{Théorème}
\newtheorem{notation}[theorem]{Notation}
  \title[Construire un astrolabe]{Construire un astrolabe}
  
  \author{Francesco De Comité}\institute{Ex-Université de Lille \\ Faculté des Sciences et Technologies}
%\titlegraphic{\includegraphics[width=3.5cm]{images/logo.jpg}\hfill}

%\titlegraphic{\ \vfill\vfill\hfill\includegraphics[width=3.5cm]{images/logo.jpg}\hfill\hfill\hfill\hfill \raisebox{0.25cm}{\includegraphics[width=1.1cm]{images/logoCNRS.jpg}}\hfill\hfill}


%\logo{\includegraphics[height=0.5cm]{lille1.png}}

\setbeamertemplate{part page}{
        \begin{beamercolorbox}[sep=8pt,center,wd=\textwidth]{part title}
            \usebeamerfont{part title}\insertpart\par
        \end{beamercolorbox}
}


%%%%%%%%%%%%%%%

\setbeamertemplate{part page}{
        \begin{beamercolorbox}[sep=8pt,center,wd=\textwidth, rounded=true,shadow=true]{part title}
            \usebeamerfont{part title}\insertpart\par
        \end{beamercolorbox}
}

%\definecolor{Universite}{RGB}{181,0,103}
\definecolor{Espace}{RGB}{0,37,115}

   \setbeamercolor{item projected}{bg=Espace}

\setbeamercolor{title}{bg=Espace}
\setbeamercolor{part title}{bg=Espace}
\setbeamercolor{frametitle}{bg=Espace}
%\setbeamercolor{structure}{bg=Universite}
\setbeamercolor{block title}{bg=Espace}
%\setbeamercolor{title}{fg=white,bg=!181!0!103}

\setbeamercolor{section in head/foot}{bg=Espace}

\makeatletter
\setbeamertemplate{footline}
{
  \leavevmode%
  \hbox{%
  \begin{beamercolorbox}[wd=.25\paperwidth,ht=2.25ex,dp=1ex,center]{section in head/foot}%
    \usebeamerfont{author in head/foot}\insertshortauthor
  \end{beamercolorbox}%
  \begin{beamercolorbox}[wd=.60\paperwidth,ht=2.25ex,dp=1ex,center]{section in head/foot}%
    \usebeamerfont{title in head/foot}\insertshorttitle
  \end{beamercolorbox}%
  \begin{beamercolorbox}[wd=.15\paperwidth,ht=2.25ex,dp=1ex,right]{section in head/foot}%
   % \usebeamerfont{date in head/foot}\insertshortdate{}\hspace*{2em}
    %\insertframenumber{} / \inserttotalframenumber\hspace*{2ex} 
	%\hfill\raisebox{-0.11cm}{\includegraphics[width=0.7cm]{images/logo.jpg}}
  \end{beamercolorbox}}%
  \vskip0pt%
}
\makeatother


%%%%%%%%%%%%%%

\begin{document}

%%%%%%%%%%%%%%%%%%%%%%%%%%%%
 \begin{frame}
\titlepage
\note[item]{bla bla}
   \end{frame}
%%%%%%%%%%%%%%%%%%%%%%%%%%%%%%%

%%%%%%%%%%%%%%%%%%%%%%%%%%%%%%%%%%%
  
 \begin{frame}\frametitle{Généralités}

\begin{block}{Qui suis-je ? }
\begin{itemize}
\item Maître de conférences en informatique (retraité) \\ à l'Université de Lille
\item Enseignements en programmation, maths pour l'informatique, logique, intelligence artificielle. 
\item Recherche en IA, puis en Art, Mathématiques, Informatique. 
\item \begin{block}{Donner une réalité à des objets mathématiques. }
	\begin{itemize}
	\item Logiciels adéquats (Blender, Rhino, Inkscape)
	\item Langages de description (SVF, STL)
	\item Outils de conception : imprimantes 3D, découpe laser, atelier {\it ``standard''}.
	\end{itemize}
	\end{block}
\end{itemize}
\end{block}

   \end{frame}


%%%%%%%%%%%%%%%%%%%%%%%%%%%%%%%%%%%
  
 \begin{frame}\frametitle{Pourquoi construire un astrolabe ?}
 
 \begin{block}{Parce qu'on peut !}
 \begin{itemize}
 \item Des documents très bien écrits (Michel Dumas).
 \item Des logiciels.
 \item Des machines et des lieux : fablabs. 
 \item Des gens. 
 \end{itemize}
 
 
 
 \end{block}
 
\end{frame}


\begin{frame}\frametitle{N'hésitez pas à interrompre !}
\begin{center}
\pgfuseimage{ask}
\end{center}
{\hfill \small Créé grace à Dall-E}
   \end{frame}


      
   

\begin{frame}\frametitle{Toutes les infos dans le GIT}
\begin{block}{GIT minimal}
\begin{itemize}
\item GIT (Global Information tracker) permet un travail coopératif et le suivi de
versions de projets (informatiques). 
\item GIT permet de rendre des fichiers, programmes, etc... disponibles pour plusieurs utilisateurs.
\item Utilisé aussi par les {\it makers} pour diffuser leur projet. 
\item Permet un retour des utilisateurs (critiques, améliorations \dots)
\end{itemize}
\end{block}
   \end{frame}
   
   \begin{frame}\frametitle{With a little help from my friends}
\begin{block}{Des documents très utiles}
\begin{itemize}
\item {\em Réalisation d'un astrolabe} Michel Dumas. 
\item {\em Les secrets de l'astrolabe} Ya{\"e}l Nazé.
\item Ces documents sont dans le dépôt GIT. 
\end{itemize}
\end{block}

   \end{frame}

\begin{frame}\frametitle{GIT en pratique -1}
{\small \tt https://github.com/francescodecomite/Construire-un-astrolabe}
{\tt https://tinyurl.com/39dbhsa5}
\begin{center}
\pgfuseimage{qrcode}
\end{center}
\vfill
\ 

\end{frame}

\begin{frame}\frametitle{GIT en pratique -- Simples consommateurs }
Simples consommateurs
\begin{center}
\hglue -1cm \pgfuseimage{githubconso1}
\end{center}

\end{frame}

\begin{frame}\frametitle{GIT en pratique -- Simples consommateurs}
\begin{center}
\hglue -1cm \pgfuseimage{fichierssvg}
\end{center}

   \end{frame}
   
   \begin{frame}\frametitle{GIT en pratique -- Pour les pros}
\begin{center}
\hglue -2cm \pgfuseimage{githubpro1}
\end{center}

   \end{frame}
   
   
   


\begin{frame}\frametitle{Le déroulé}
\begin{block}{Le plan}
\begin{itemize}
\item C'est quoi un astrolabe ?
\item Les différents composants. 
\item Qu'est-ce qu'on peut faire avec ? 
\item Comment on calcule les différentes parties ? 
\item Comment on programme ? 
\item Comment on construit ?

\end{itemize}
\end{block}

   \end{frame}
   
\begin{frame}\frametitle{Un exemple d'astrolabe}
\begin{center}
\pgfuseimage{astrolabe1}
\end{center}
{\hfill Crédits : Landesmuseum Württemberg, Stuttgart}



   \end{frame}
   
   \begin{frame}\frametitle{Un exemple d'astrolabe}
\begin{center}
\pgfuseimage{astrolabe2}
\end{center}
{\hfill Crédits : Landesmuseum Württemberg, Stuttgart}
   \end{frame}
   
\begin{frame}\frametitle{}

   \end{frame}
   
\begin{frame}\frametitle{}

   \end{frame}
 \end{document}

 \begin{frame}\frametitle{Carrière}
 \begin{itemize}
 \item Nommé maître de conférences en 1984
 \end{itemize}
\begin{block}{Domaines de recherche  }
\begin{itemize}
\item Informatique théorique ({\em systèmes de réécriture})
\item Intelligence artificielle (apprentissage automatique : réseaux de neurones \dots)
\item Art et ordinateur.
\end{itemize}
\end{block}


   \end{frame}

 \begin{frame}\frametitle{Art et ordinateur}
 \begin{block}{La démarche}
 \begin{itemize}
 \item Représenter des concepts mathématiques sous formes de dessins, de scènes, d'objets. 
 \item Utiliser des outils informatiques si possibles libres, open-source, gratuits. 
 \item Expliquer la méthode utilisée, diffuser ce que j'ai appris. 
 \end{itemize}
\end{block}
\end{frame}

\begin{frame}\frametitle{Outils}

   
\begin{block}{Les logiciels}
\begin{itemize}
\item Gimp : Manipulation d'images ($\approx$ Photoshop)
\item Povray : Création de scènes. 
\item Blender : Modélisation 3D. Produit dessins et objets 3D
\item Rhino + Grasshopper: Comme Blender, mais payant (versions d'évaluations limitées)
\item Langages de programmation :Java, Python
\item Inkscape : découpe laser. 
\end{itemize}
\end{block}
\end{frame}



\item Pour obtenir de belles images, il faut un ordinateur rapide et un programme efficace. 
\end{itemize}
\end{block}
   \end{frame}
   
\begin{frame}\frametitle{Mandelbulb : une généralisation en 3D}
\begin{center}
\pgfuseimage{bulb1}
\end{center}
   \end{frame}
   
   \begin{frame}\frametitle{Mandelbulb : une généralisation en 3D}
\begin{center}
\pgfuseimage{bulb2}
\end{center}
   \end{frame}
   
   \begin{frame}\frametitle{Mandelbulb : une généralisation en 3D}
\begin{center}
\pgfuseimage{bulb3}
\end{center}
   \end{frame}
   
   
   \begin{frame}\frametitle{Suite du programme et pistes}
   \begin{block}{Cercles tangents}
   \begin{itemize}
   \item En reprenant le théorème de Descartes, on peut penser à un procédé de création d'un ensemble de cercles tangents. 
   \item Si on utilise la version complexe, on peut penser à une installation géante. 
   \end{itemize}
   
  
   \end{block}


   \end{frame}


    \begin{frame}\frametitle{Perspectives et applications possibles}
\begin{center}
\pgfuseimage{denevan}

\end{center}
\hfill{ Jim Denevan}
\note[item]{Jim Denevan : j'ai déjà essayé de faire ça à la plage, mais il faut quand même avoir pas mal de metier\dots}
   \end{frame}
   








































\end{document}

\begin{frame}\frametitle{}

   \end{frame}
   
\begin{frame}\frametitle{}

   \end{frame}
   
\begin{frame}\frametitle{}

   \end{frame}
   
\begin{frame}\frametitle{}

   \end{frame}
   

