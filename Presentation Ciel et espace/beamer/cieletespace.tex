
\documentclass{beamer}
%\documentclass[handout]{beamer}
\usepackage{pgfpages}
%\setbeameroption{show notes on second screen=left}
%\setbeameroption{hide notes}
%\setbeameroption{show only  notes}

\usepackage{xcolor}
\usepackage[utf8]{inputenc}
\usepackage[francais]{babel}
\usepackage{multimedia}
%comment
\setbeamertemplate{navigation symbols}{} 
 \usetheme{Warsaw}
\setbeamercovered{highly dynamic}

\usepackage{multimedia}

%\usepackage{pgfpages}
%\mode<handout>{\pgfpagesuselayout{4 on 1}[a4paper,landscape,border shrink=5mm]}
\mode<handout>{\pgfpagesuselayout{8 on 1}[a4paper,border shrink=5mm]}
\mode<handout>{\setbeamercolor{background canvas}{bg=black!5}}


\pgfdeclareimage[height=0.9\textheight]{ask}{images/ask}
\pgfdeclareimage[height=0.9\textheight]{github}{images/github}
\pgfdeclareimage[height=0.9\textheight]{qrcode}{images/qrcode}
\pgfdeclareimage[height=0.9\textheight]{githubconso1}{images/githubconso1}
\pgfdeclareimage[height=0.9\textheight]{fichierssvg}{images/fichierssvg}
\pgfdeclareimage[height=0.9\textheight]{githubpro1}{images/githubpro1}
\pgfdeclareimage[height=0.85\textheight]{astrolabe1}{images/astrolabe1}
\pgfdeclareimage[height=0.85\textheight]{astrolabe2}{images/astrolabe2}
\pgfdeclareimage[height=0.9\textheight]{davant}{images/descriptionavant.png}
\pgfdeclareimage[height=0.9\textheight]{tympanannote}{images/tympan44annote}
\pgfdeclareimage[height=0.9\textheight]{reteannote}{images/reteannote}
\pgfdeclareimage[height=0.9\textheight]{10octobre1}{images/10octobre1}
\pgfdeclareimage[height=0.9\textheight]{10octobre2}{images/10octobre2}
\pgfdeclareimage[height=0.9\textheight]{10octobre3}{images/10octobre3}
\pgfdeclareimage[height=0.9\textheight]{10octobre4}{images/10octobre4}
\pgfdeclareimage[height=0.9\textheight]{10octobre5}{images/10octobre5}
\pgfdeclareimage[height=0.9\textheight]{10octobre6}{images/10octobre6}
\pgfdeclareimage[height=0.9\textheight]{facearriere}{images/facearriere}
\pgfdeclareimage[height=0.9\textheight]{arriereannote}{images/arriereannote}
\pgfdeclareimage[width=1\textwidth]{equation}{images/equation}
\pgfdeclareimage[height=0.9\textheight]{analemme}{images/analemme}
\pgfdeclareimage[height=0.9\textheight]{exempleequation}{images/exempleequation}
\pgfdeclareimage[height=0.9\textheight]{alidade}{images/alidade}
\pgfdeclareimage[height=0.9\textheight]{arriere}{images/arriere}
\pgfdeclareimage[height=0.9\textheight]{ostenseur}{images/ostenseur}
\pgfdeclareimage[height=0.8\textheight]{tympan50}{images/tympan50}
\pgfdeclareimage[height=0.9\textheight]{retedebase}{images/retedebase}
\pgfdeclareimage[height=0.9\textheight]{retefinal}{images/retefinal}
\pgfdeclareimage[height=1.3\textheight]{calcul}{images/calcul}
\pgfdeclareimage[height=0.9\textheight]{chou}{images/chou}
\pgfdeclareimage[height=0.9\textheight]{fougere}{images/fougere}
\pgfdeclareimage[height=0.9\textheight]{fern}{images/fern}
\pgfdeclareimage[height=0.9\textheight]{foug}{images/foug}
\pgfdeclareimage[width=\textwidth]{cantor}{images/cantor}
\pgfdeclareimage[height=0.9\textheight]{koch}{images/koch}
\pgfdeclareimage[height=0.9\textheight]{hilbert}{images/hilbert}
\pgfdeclareimage[height=0.8\textheight]{mandelbrot}{images/mandelbrot}
\pgfdeclareimage[height=0.8\textheight]{brooksmatelski1978}{images/Mandel}
\pgfdeclareimage[height=0.9\textheight]{m1}{images/m1}
\pgfdeclareimage[height=0.9\textheight]{m2}{images/m2}
\pgfdeclareimage[height=0.9\textheight]{m3}{images/m3}
\pgfdeclareimage[height=0.9\textheight]{m4}{images/m4}
\pgfdeclareimage[height=0.9\textheight]{bulb1}{images/bulb1}
\pgfdeclareimage[height=0.9\textheight]{bulb2}{images/bulb2}
\pgfdeclareimage[height=0.9\textheight]{bulb3}{images/bulb3}
\pgfdeclareimage[height=0.8\textheight]{denevan}{images/denevan}



\newtheorem{fdefinition}[theorem]{Définition}
\newtheorem{Remarque}[theorem]{Remarque}
\newtheorem{Remarques}[theorem]{Remarques}
%\newtheorem{Theoreme}[theorem]{Théorème}
\newtheorem{notation}[theorem]{Notation}
  \title[Construire un astrolabe]{Construire un astrolabe}
  
  \author{Francesco De Comité}\institute{Ex-Université de Lille \\ Faculté des Sciences et Technologies}
\titlegraphic{\includegraphics[width=3.5cm]{images/Logo_fabricarium.png}\hfill}

%\titlegraphic{\ \vfill\vfill\hfill\includegraphics[width=3.5cm]{images/Logo_fabricarium.png}\hfill\hfill}


%\logo{\includegraphics[height=0.5cm]{lille1.png}}

\setbeamertemplate{part page}{
        \begin{beamercolorbox}[sep=8pt,center,wd=\textwidth]{part title}
            \usebeamerfont{part title}\insertpart\par
        \end{beamercolorbox}
}


%%%%%%%%%%%%%%%

\setbeamertemplate{part page}{
        \begin{beamercolorbox}[sep=8pt,center,wd=\textwidth, rounded=true,shadow=true]{part title}
            \usebeamerfont{part title}\insertpart\par
        \end{beamercolorbox}
}

%\definecolor{Universite}{RGB}{181,0,103}
\definecolor{Espace}{RGB}{0,37,115}

   \setbeamercolor{item projected}{bg=Espace}

\setbeamercolor{title}{bg=Espace}
\setbeamercolor{part title}{bg=Espace}
\setbeamercolor{frametitle}{bg=Espace}
%\setbeamercolor{structure}{bg=Universite}
\setbeamercolor{block title}{bg=Espace}
%\setbeamercolor{title}{fg=white,bg=!181!0!103}

\setbeamercolor{section in head/foot}{bg=Espace}

\makeatletter
\setbeamertemplate{footline}
{
  \leavevmode%
  \hbox{%
  \begin{beamercolorbox}[wd=.25\paperwidth,ht=2.25ex,dp=1ex,center]{section in head/foot}%
    \usebeamerfont{author in head/foot}\insertshortauthor
  \end{beamercolorbox}%
  \begin{beamercolorbox}[wd=.60\paperwidth,ht=2.25ex,dp=1ex,center]{section in head/foot}%
    \usebeamerfont{title in head/foot}\insertshorttitle
  \end{beamercolorbox}%
  \begin{beamercolorbox}[wd=.15\paperwidth,ht=2.25ex,dp=1ex,center]{section in head/foot}%
    %\usebeamerfont{date in head/foot}\insertshortdate{}\hspace*{2em}
     \insertframenumber/\inserttotalframenumber\hspace*{0.01ex} 
	\hfill\raisebox{-0.05cm}{\includegraphics[width=1.1cm]{images/Logo_fabricarium.png}}
  \end{beamercolorbox}}%
  \vskip0pt%
}
\makeatother


%%%%%%%%%%%%%%

\begin{document}

%%%%%%%%%%%%%%%%%%%%%%%%%%%%
 \begin{frame}
\titlepage
\note[item]{bla bla}
   \end{frame}
%%%%%%%%%%%%%%%%%%%%%%%%%%%%%%%

%%%%%%%%%%%%%%%%%%%%%%%%%%%%%%%%%%%
  
 \begin{frame}\frametitle{Généralités}

\begin{block}{Qui suis-je ? }
\begin{itemize}
\item Maître de conférences en informatique (retraité) \\ à l'Université de Lille
\item Enseignements en programmation, maths pour l'informatique, logique, intelligence artificielle. 
\item Recherche en IA, puis en Art, Mathématiques, Informatique. 
\item \begin{block}{Donner une réalité à des objets mathématiques. }
	\begin{itemize}
	\item Logiciels adéquats (Blender, Rhino, Inkscape)
	\item Langages de description (SVF, STL)
	\item Outils de conception : imprimantes 3D, découpe laser, atelier {\it ``standard''}.
	\end{itemize}
	\end{block}
\end{itemize}
\end{block}

   \end{frame}


%%%%%%%%%%%%%%%%%%%%%%%%%%%%%%%%%%%
  
 \begin{frame}\frametitle{Pourquoi construire un astrolabe ?}
 
 \begin{block}{Parce qu'on peut !}
 \begin{itemize}
 \item Des documents très bien écrits (Michel Dumas).
 \item Des logiciels.
 \item Des machines et des lieux : fablabs. 
 \item Des gens. 
 \end{itemize}
 
 
 
 \end{block}
 
\end{frame}


\begin{frame}\frametitle{N'hésitez pas à interrompre !}
\begin{center}
\pgfuseimage{ask}
\end{center}
{\hfill \small Créé grace à Dall-E}
   \end{frame}


      
   

\begin{frame}\frametitle{Toutes les infos dans le GIT}
\begin{block}{GIT minimal}
\begin{itemize}
\item GIT (Global Information tracker) permet un travail coopératif et le suivi de
versions de projets (informatiques). 
\item GIT permet de rendre des fichiers, programmes, etc... disponibles pour plusieurs utilisateurs.
\item Utilisé aussi par les {\it makers} pour diffuser leur projet. 
\item Permet un retour des utilisateurs (critiques, améliorations \dots)
\end{itemize}
\end{block}
   \end{frame}
   
   \begin{frame}\frametitle{With a little help from my friends}
\begin{block}{Des documents très utiles}
\begin{itemize}
\item {\em Réalisation d'un astrolabe} Michel Dumas. 
\item {\em Les secrets de l'astrolabe} Ya{\"e}l Nazé.
\item Ces documents sont dans le dépôt GIT. 
\end{itemize}
\end{block}

   \end{frame}

\begin{frame}\frametitle{GIT en pratique -1}
{\small \tt https://github.com/francescodecomite/Construire-un-astrolabe}
{\tt https://tinyurl.com/39dbhsa5}
\begin{center}
\pgfuseimage{qrcode}
\end{center}
\vfill
\ 

\end{frame}

\begin{frame}\frametitle{GIT en pratique -- Simples consommateurs }
Simples consommateurs
\begin{center}
\hglue -1cm \pgfuseimage{githubconso1}
\end{center}

\end{frame}

\begin{frame}\frametitle{GIT en pratique -- Simples consommateurs}
\begin{center}
\hglue -1cm \pgfuseimage{fichierssvg}
\end{center}

   \end{frame}
   
   \begin{frame}\frametitle{GIT en pratique -- Pour les pros}
\begin{center}
\hglue -2cm \pgfuseimage{githubpro1}
\end{center}

   \end{frame}
   
   
   


\begin{frame}\frametitle{Le déroulé}
\begin{block}{Le plan}
\begin{itemize}
\item C'est quoi un astrolabe ?
\item Les différents composants. 
\item Qu'est-ce qu'on peut faire avec ? 
\item Comment on calcule les différentes parties ? 
\item Comment on programme ? 
\item Comment on construit ?

\end{itemize}
\end{block}

   \end{frame}
   
\begin{frame}\frametitle{Un exemple d'astrolabe : la face avant}
\begin{center}
\pgfuseimage{astrolabe1}
\end{center}
{\hfill Crédits : Landesmuseum Württemberg, Stuttgart}



   \end{frame}
   
   \begin{frame}\frametitle{Un exemple d'astrolabe : la face arrière}
\begin{center}
\pgfuseimage{astrolabe2}
\end{center}
{\hfill Crédits : Landesmuseum Württemberg, Stuttgart}
   \end{frame}
   

\begin{frame}\frametitle{La face avant en détail}
\begin{center}
\pgfuseimage{davant}
\end{center}

   \end{frame}
 
   
\begin{frame}\frametitle{Face avant. Le tympan}
\begin{center}
\pgfuseimage{tympanannote}
\end{center}
   \end{frame}


\begin{frame}\frametitle{Face avant. Le Rete}
\begin{center}
\pgfuseimage{reteannote}
\end{center}
   \end{frame} 



\begin{frame}\frametitle{La face arrière en détail}
\begin{center}
\pgfuseimage{arriereannote}
\end{center}
   \end{frame}
   
   \begin{frame}\frametitle{Face arrière complète}
\begin{center}
\pgfuseimage{facearriere}
\end{center}
   \end{frame}
   
\begin{frame}\frametitle{Un survol des fonctions}
\begin{block}{Face avant}
\begin{itemize}
\item Carte du ciel.
\item Quels sont l'heure et l'azimuth du lever de Soleil à une date donnée ? 
\item Quelle est la hauteur maximale du Soleil à une latitude donnée ? 
\item A quelle heure un étoile atteint-elle une altitude donnée, connaissant le jour et la latitude ? 
\item Que valent l'ascension droite et la déclinaison d'un astre ? 
\item {\em Pour ces quatre dernières questions, cf Y. Nazé}
\end{itemize}


\end{block}

   \end{frame}
   \begin{frame}\frametitle{Carte du ciel}
\begin{center}
\pgfuseimage{10octobre1}
\end{center}
   \end{frame}
   
   
      \begin{frame}\frametitle{Carte du ciel}
\begin{center}
\pgfuseimage{10octobre2}
\end{center}
   \end{frame}
 
 
    
        \begin{frame}\frametitle{Carte du ciel}
\begin{center}
\pgfuseimage{10octobre6}
\end{center}
   \end{frame}
   
   
\begin{frame}\frametitle{Carte du ciel : vérification}
\begin{center}
\pgfuseimage{10octobre5}
\end{center}
   \end{frame}
   \begin{frame}\frametitle{Un survol des fonctions}
   \begin{block}{La face arrière}
   \begin{itemize}
   \item Hauteur d'un bâtiment.
   \item Calcul des heures inégales. 
   \item Equation du temps
   \item Hauteur d'un astre au-dessus de l'horizon. 
   
   \end{itemize}
   
   \end{block}

   \end{frame}
   
\begin{frame}\frametitle{Ma préférée}
\begin{block}{L'équation du temps}
\begin{itemize}
\item A midi, il est rarement midi\dots
\begin{itemize}
\item Parce que l'orbite de la terre est une ellipse. 
\item Parce la terre est inclinée sur son orbite.
\end{itemize}
\item Le soleil peut être en avance ou en retard de 15 minutes sur son horaire. 
\end{itemize}

\end{block}

   \end{frame}
   \begin{frame}\frametitle{Equation du temps}
\begin{center}
\pgfuseimage{equation}
\end{center}
\hfill \small Crédit Wikipédia
   \end{frame}
   
\begin{frame}\frametitle{Equation du temps}
\begin{center}
\pgfuseimage{analemme}
\end{center}

\hfill \small Crédit Rhodri Evans
   \end{frame}
   \begin{frame}\frametitle{Equation du temps}
\begin{center}
\pgfuseimage{exempleequation}
\end{center}
   \end{frame}

\begin{frame}[fragile]\frametitle{Construire l'astrolabe}
\begin{verbatim}
    

if __name__=="__main__":
   #alidade(R/2)  
   #ostenseur(R/2)
   #dos(R/2)
   tympan(R/2)
   tympanseul(R/2)
   #rete(R/2)
  
  

\end{verbatim}

   \end{frame}
   
   \begin{frame}\frametitle{Les fichiers prêts à découper.  Alidade}
\begin{center}
\pgfuseimage{alidade}
\end{center}
   \end{frame}
   
\begin{frame}\frametitle{Les fichiers prêts à découper.  Ostenseur}
\begin{center}
\pgfuseimage{ostenseur}
\end{center}
   \end{frame}
   
\begin{frame}\frametitle{Les fichiers prêts à découper.  Tympan50}
\begin{center}
\pgfuseimage{tympan50}
\end{center}
   \end{frame}
   

\begin{frame}\frametitle{Les fichiers prêts à découper.  Rete de base}
\begin{center}
\pgfuseimage{retedebase}
\end{center}
   \end{frame}
   
\begin{frame}\frametitle{Les fichiers prêts à découper.  Rete final}
\begin{center}
\pgfuseimage{retefinal}
\end{center}
   \end{frame}
   
\begin{frame}\frametitle{}

   \end{frame}
   

\begin{frame}\frametitle{}

   \end{frame}
   
\begin{frame}\frametitle{}

   \end{frame}
   
\begin{frame}\frametitle{}

   \end{frame}
   

\begin{frame}\frametitle{}

   \end{frame}
   
\begin{frame}\frametitle{}

   \end{frame}
   
\begin{frame}\frametitle{}

   \end{frame}
   


   
   \end{document}
\begin{frame}\frametitle{}

   \end{frame}
   
\begin{frame}\frametitle{}

   \end{frame}
   
\begin{frame}\frametitle{}

   \end{frame}
   

